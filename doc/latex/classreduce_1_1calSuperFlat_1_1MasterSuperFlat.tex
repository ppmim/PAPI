\section{reduce::cal\-Super\-Flat::Master\-Super\-Flat Class Reference}
\label{classreduce_1_1calSuperFlat_1_1MasterSuperFlat}\index{reduce::calSuperFlat::MasterSuperFlat@{reduce::calSuperFlat::MasterSuperFlat}}
\subsection*{Public Member Functions}
\begin{CompactItemize}
\item 
def {\bf\_\-\_\-init\_\-\_\-}
\item 
def {\bfcreate\-Master}
\end{CompactItemize}


\subsection{Detailed Description}


\footnotesize\begin{verbatim}
\brief Class used to build and manage a master calibration super flat from a list of night SCIENCE frames
\par Class:
    MasterSuperFlat
\par Purpose:
    Create a master normalized night Super Flat Field
\par Description:
    
    1. Check the TYPE(science) and FILTER and EXPTIME of each frame
       If any frame on list missmatch the FILTER,TYPE,EXPTIME,NCOADDS then the master super-flat will be aborted
    
    2. Make the combine of Flat frames scaling by 'mode'
    
    3. If required, we subtract a proper MASTER_DARK using the provided master_dark file (optional)

    4. If required, normalize the super-flat dividing by the mean value (default True)
    
\par Language:
    PyRaf
\param data
    A list of science frames
\param bpm
    Input bad pixel mask or NULL
\param mdark
    Master dark to subtract (optional)
\retval median
    When all goes well
\retval 0
    If no error
\author
    JMIbannez, IAA-CSIC
\end{verbatim}
\normalsize
 



\subsection{Member Function Documentation}
\index{reduce::calSuperFlat::MasterSuperFlat@{reduce::cal\-Super\-Flat::Master\-Super\-Flat}!__init__@{\_\-\_\-init\_\-\_\-}}
\index{__init__@{\_\-\_\-init\_\-\_\-}!reduce::calSuperFlat::MasterSuperFlat@{reduce::cal\-Super\-Flat::Master\-Super\-Flat}}
\subsubsection{\setlength{\rightskip}{0pt plus 5cm}def reduce::cal\-Super\-Flat::Master\-Super\-Flat::\_\-\_\-init\_\-\_\- ( {\em self},  {\em input\_\-files},  {\em output\_\-filaname} = {\tt \char`\"{}/tmp/msflat.fits\char`\"{}},  {\em master\_\-dark} = {\tt None},  {\em output\_\-dir} = {\tt \char`\"{}/tmp\char`\"{}},  {\em normal} = {\tt True},  {\em force} = {\tt False})}\label{classreduce_1_1calSuperFlat_1_1MasterSuperFlat_37a01235bdf44faf9dae9e995a83bfb8}




\footnotesize\begin{verbatim}
\brief Init the object
\param input_file - data science files, non dark subtracted
\param output_filename - output master super flat produced 
\param master_dark - master dark to be subtracted to each science frame
\param working_dir - dir used for computations and intermediate files
\end{verbatim}
\normalsize
 \index{reduce::calSuperFlat::MasterSuperFlat@{reduce::cal\-Super\-Flat::Master\-Super\-Flat}!createMaster@{createMaster}}
\index{createMaster@{createMaster}!reduce::calSuperFlat::MasterSuperFlat@{reduce::cal\-Super\-Flat::Master\-Super\-Flat}}
\subsubsection{\setlength{\rightskip}{0pt plus 5cm}def reduce::cal\-Super\-Flat::Master\-Super\-Flat::create\-Master ( {\em self})}\label{classreduce_1_1calSuperFlat_1_1MasterSuperFlat_b04bacd62ca9d794c172cc7ea9b09027}




\footnotesize\begin{verbatim}
\brief Create a master SUPER FLAT from a list for science files
\end{verbatim}
\normalsize
 

The documentation for this class was generated from the following file:\begin{CompactItemize}
\item 
reduce/cal\-Super\-Flat.py\end{CompactItemize}
