\section{reduce::cal\-Non\-Lienarity::Non\-Linearity\-Model Class Reference}
\label{classreduce_1_1calNonLienarity_1_1NonLinearityModel}\index{reduce::calNonLienarity::NonLinearityModel@{reduce::calNonLienarity::NonLinearityModel}}
\subsection*{Public Member Functions}
\begin{CompactItemize}
\item 
def \textbf{\_\-\_\-init\_\-\_\-}\label{classreduce_1_1calNonLienarity_1_1NonLinearityModel_f9caffb3776bff2f8dc76465e7677b10}

\item 
def {\bfcreate\-Dark\-Model}
\end{CompactItemize}


\subsection{Detailed Description}


\footnotesize\begin{verbatim}
\brief Class used to build the Non-linearity model for the detectors

\par Class:
     NonLinearityModel   
\par Purpose:
    Create a Non-Linearity Model from a serie of sky images
\par Description:

\par Language:
    PyRaf
\param input_data
    A list of sky files
\param
\retval 0
    If no error, 
    a0 = bias level
    a1 = quantum efficiency or sensitivity
    a2 = the non-linearity or saturation
\author
    JMIbannez, IAA-CSIC
\end{verbatim}
\normalsize
 



\subsection{Member Function Documentation}
\index{reduce::calNonLienarity::NonLinearityModel@{reduce::cal\-Non\-Lienarity::Non\-Linearity\-Model}!createDarkModel@{createDarkModel}}
\index{createDarkModel@{createDarkModel}!reduce::calNonLienarity::NonLinearityModel@{reduce::cal\-Non\-Lienarity::Non\-Linearity\-Model}}
\subsubsection{\setlength{\rightskip}{0pt plus 5cm}def reduce::cal\-Non\-Lienarity::Non\-Linearity\-Model::create\-Dark\-Model ( {\em self})}\label{classreduce_1_1calNonLienarity_1_1NonLinearityModel_3cef1c99ebd5d40780ee75f6b55b7a6e}




\footnotesize\begin{verbatim}
\brief Create a master DARK model from the dark file list
\end{verbatim}
\normalsize
 

The documentation for this class was generated from the following file:\begin{CompactItemize}
\item 
reduce/cal\-Non\-Lienarity.py\end{CompactItemize}
